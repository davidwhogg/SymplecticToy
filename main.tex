% Copyright 2025 the author(s).\

% style notes
% -----------
% - write such that physicists, mathematicians, and statisticians would all be happy.
% - all equations get numbers; no exceptions. Use the align environment.
% - the multiply operator is \,
% - etc.

\documentclass{article}
\usepackage[letterpaper]{geometry}
\usepackage{amsmath, amsfonts}

% typesetting
\setlength{\textheight}{8.70in}
\setlength{\textwidth}{5.00in}
\setlength{\oddsidemargin}{3.25in}\addtolength{\oddsidemargin}{-0.5\textwidth}
\setlength{\topmargin}{-0.40in}
\frenchspacing\sloppy\sloppypar\raggedbottom

% math macros
\newcommand{\dd}{\mathrm{d}}
\DeclareMathOperator*{\argmin}{argmin}
\newcommand{\norm}[1]{\left\lVert#1\right\rVert_2^2}

% units macros
\newcommand{\unit}[1]{\mathrm{#1}}
\newcommand{\m}{\unit{m}}
\newcommand{\s}{\unit{s}}
\newcommand{\ps}{\s^{-1}}
\newcommand{\mps}{\m\,\ps}
\newcommand{\mmps}{\m^2\,\ps}

\title{\bfseries%
A toy problem for symplectic maps}
\author{The participants of \textsl{Data And Geometry}}
\date{2025 January}

\begin{document}

\maketitle

\section{One-dimensional problem}
You are given $n$ one-dimensional positions $x_i$ (units of $\m$) and one-dimensional velocities $v_i$ (units of $\mps$) corresponding to times $t_i$ (units of $\s$), with $1\leq i\leq n$.
These points are $n$ snapshots at times $t_i$ of a particle in phase space on an orbit.
Our goal is to find an area-preserving map $T_W$ from $x, v$ to new coordinates $\xi, \eta$ such that
\begin{align}
    \begin{bmatrix}\xi^{(W)}_i \\ \eta^{(W)}_i\end{bmatrix} &= T_W\left(\begin{bmatrix}x_i \\ v_i\end{bmatrix}\right) ~,
\end{align}
where the function $T_W$ is presumably extremely nonlinear in its inputs.
The new coordinates ought to lie (nearly) on an ellipse of the form
\begin{align}
    \begin{bmatrix}\xi^{(J)}_i \\ \eta^{(J)}_i\end{bmatrix} &= \begin{bmatrix}\sqrt{2\,J\,\omega^{-1}}\,\sin(\omega\,t_i+\phi) \\ \sqrt{2\,J\,\omega}\,\cos(\omega\,t_i+\phi)\end{bmatrix}
\end{align}
for some values of $J$ (units of $\mmps$), $\omega$ (units of $\ps$), and $\phi$ (dimensionless).\footnote{Side question: Do we need to have a $\phi$ variable? Or can we set $\phi=0$ without loss of generality?}
The area-preserving map will have parameters or weights $W$ (which, in practice, will be a large structured blob of numbers, some of which may have units).

We are going to solve this problem by minimizing a cost function with respect to the parameters $W,J,\omega,\phi$.
This could be something ``discriminative'' like
\begin{align}
    \hat{W},\hat{J},\hat{\omega},\hat{\phi} &\leftarrow \argmin_{W,J,\omega,\phi} \sum_{i=1}^n\norm{\begin{bmatrix}\xi^{(W)}_i \\ \eta^{(W)}_i\end{bmatrix} - \begin{bmatrix}\xi^{(J)}_i \\ \eta^{(J)}_i\end{bmatrix}} ~.
\end{align}
Or this could be something ``generative'' like
\begin{align}
    \hat{W},\hat{J},\hat{\omega},\hat{\phi} &\leftarrow \argmin_{W,J,\omega,\phi} \sum_{i=1}^n\norm{\begin{bmatrix}x_i \\ v_i\end{bmatrix} - T_W^{-1}(\begin{bmatrix}\xi^{(J)}_i \\ \eta^{(J)}_i\end{bmatrix})} ~,
\end{align}
where $T_W^{-1}$ is the inverse of the map.
In principle the generative direction is more scientifically responsible, if we think of the $x_i, y_i$ as being ``data.''
Since our goal is $T_W$ and not (particularly) $T_W^{-1}$, which we choose depends on how easy it is to get inverses.\footnote{Presumably getting inverses is either very hard or else must be framed as a separate learning problem?}

\subsection{An Attempt to Show Hogg's Conjecture}

We want to show that, under some regularity conditions on the original orbits in the universe, there is a symplectic map $f : R^2 \to R^2$ such that the orbits become elliptical with a fixed angular frequency. 
Let's consider a single particle. 

To show this, let's write out a Hamiltonian to describe the original orbits
\begin{align}
    \frac{\dd x}{\dd t} &= - \frac{\partial \mathcal{H}}{\partial v}, \quad \frac{d v}{dt} = \frac{\partial \mathcal{H}}{\partial x}. 
\end{align}
We want to get a closed-form ODE in terms of the new phase parameters $f_1, f_2$ (not sure how the units come to play here). So, chain rule and plugging in the above time derivatives gives us 
\begin{align}
    \frac{\dd f_1}{\dd t} &= - \frac{\partial f_1}{\partial x} \frac{\partial \mathcal{H}}{\partial v} + \frac{\partial f_1}{\partial v} \frac{\partial \mathcal{H}}{\partial x} \quad
    \frac{\dd f_2}{\dd t} = - \frac{\partial f_2}{\partial x} \frac{\partial \mathcal{H}}{\partial v} + \frac{\partial f_2}{\partial v} \frac{\partial \mathcal{H}}{\partial x}.
\end{align}

The ODE for which the solution is an elliptical orbit with a fixed angular frequency is given by 
\begin{align}
    \frac{\dd}{\dd t} \begin{bmatrix} x \\ y \end{bmatrix} &= \begin{bmatrix} 0 & a \\ -b & 0 \end{bmatrix} \begin{bmatrix} x \\ y \end{bmatrix}, \quad a\,b > 0. 
\end{align}
This is a nice linear system and admits a solution:  $x(t)=\sqrt{a} \sin(\sqrt{a\,b}\,t + \phi) $, $y(t)= \sqrt{b} \cos(\sqrt{a\,b}\,t + \phi)$; the angular frequency is $\sqrt{a\,b}$ here. $\phi$ here gives us the initial condition, but if we have data from the complete orbit, it may be dropped (?). \footnote{Teresa says: Mmh, setting $a = 2 J \omega^{-1}$ and $b = 2 J w$ to match the above parameters gives us $\sqrt{a\,b}=2\,J$ for the angular frequency. Something seems a little bit off here? Hogg says: No that's correct; $J$ is the area of the ellipse divided by $2\pi$, so this should be correct.}

To ensure that $f_1, f_2$ follows an elliptical orbit, we would like to have
\begin{align}
    - \frac{\partial f_1}{\partial x} \frac{\partial \mathcal{H}}{\partial v} + \frac{\partial f_1}{\partial v} \frac{\partial \mathcal{H}}{\partial x} &= a\,f_2, \quad
     - \frac{\partial f_2}{\partial x} \frac{\partial \mathcal{H}}{\partial v} + \frac{\partial f_2}{\partial v} \frac{\partial \mathcal{H}}{\partial x} = -b\,f_1,
\end{align}
which can also be expressed using a Poisson bracket 
\begin{align}
    [f_1, \mathcal{H}] &= a f_2, \quad [f_2, \mathcal{H}] = -b\,f_1. 
\end{align}
This calculation shows that Hogg's conjecture holds iff we can find a symplectic map $f$ such that the above condition is satisfied for some real numbers $a, b$. 

\section{Three-dimensional problem}
How does this generalize to three dimensions?
Is it as simple as
\begin{align}
    \begin{bmatrix}\boldsymbol{\xi}^{(W)}_i \\ \boldsymbol{\eta}^{(W)}_i\end{bmatrix} &= T_W(\begin{bmatrix}\boldsymbol{x}_i \\ \boldsymbol{v}_i\end{bmatrix}) \\
    \begin{bmatrix}\xi^{(J)}_{ki} \\ \eta^{(J)}_{ki}\end{bmatrix} &= \begin{bmatrix}\sqrt{2\,J_k\,\omega_k^{-1}}\,\sin(\omega_k\,t_i+\phi_k) \\ \sqrt{2\,J_k\,\omega_k}\,\cos(\omega_k\,t_i+\phi_k)\end{bmatrix} ~,
\end{align}
where $\boldsymbol{x}, \boldsymbol{v}, \boldsymbol{\xi}, \boldsymbol{\eta}$ are now three-vectors (column vectors in $\mathbb{R}^{3\times 1}$ with the same units as $x, v, \xi, \eta$ above),
and $\xi_{ki}, \eta_{ki}$ are the $k$th components of the three-vectors $1\leq k\leq 3$?

\end{document}
