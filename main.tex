% Copyright 2025 the author(s).

\documentclass{article}
\usepackage[letterpaper]{geometry}
\usepackage{amsmath}

% typesetting
\setlength{\textheight}{8.70in}
\setlength{\textwidth}{5.00in}
\setlength{\oddsidemargin}{3.25in}\addtolength{\oddsidemargin}{-0.5\textwidth}
\setlength{\topmargin}{-0.40in}
\frenchspacing\sloppy\sloppypar\raggedbottom

% math macros
\DeclareMathOperator*{\argmin}{argmin}
\newcommand{\norm}[1]{\left\lVert#1\right\rVert_2^2}

% units macros
\newcommand{\unit}[1]{\mathrm{#1}}
\newcommand{\m}{\unit{m}}
\newcommand{\s}{\unit{s}}
\newcommand{\ps}{\s^{-1}}
\newcommand{\mps}{\m\,\ps}
\newcommand{\mmps}{\m^2\,\ps}

\title{\bfseries%
A toy problem for symplectic maps}
\author{The participants of \textsl{Data And Geometry}}
\date{January 2025}

\begin{document}

\maketitle

\section{One-dimensional problem}
You are given $n$ one-dimensional positions $x_i$ (units of $\m$) and one-dimensional velocities $v_i$ (units of $\mps$) corresponding to times $t_i$ (units of $\s$), with $1\leq i\leq n$.
These points are $n$ snapshots at times $t_i$ of a particle in phase space on an orbit.
Our goal is to find an area-preserving map $T_W$ from $x, v$ to new coordinates $\xi, \eta$ such that
\begin{align}
    \begin{bmatrix}\xi^{(W)}_i \\ \eta^{(W)}_i\end{bmatrix} &= T_W(\begin{bmatrix}x_i \\ v_i\end{bmatrix}) ~,
\end{align}
where the action of $T_W$ is presumably extremely nonlinear.
The new coordinates ought to lie (nearly) on an ellipse of the form
\begin{align}
    \begin{bmatrix}\xi^{(J)}_i \\ \eta^{(J)}_i\end{bmatrix} &= \begin{bmatrix}\sqrt{J\,\omega^{-1}}\,\sin(\omega\,t_i+\phi) \\ \sqrt{J\,\omega}\,\cos(\omega\,t_i+\phi)\end{bmatrix}
\end{align}
for some values of $J$ (units of $\mmps$), $\omega$ (units of $\ps$), and $\phi$ (dimensionless).\footnote{Side question: Do we need to have a $\phi$ variable? Or can we set $\phi=0$ without loss of generality?}
The area-preserving map will have parameters or weights $W$ (which, in practice, will be a large structured blob of numbers, some of which may have units).

We are going to solve this problem by minimizing a cost function with respect to the parameters $W,J,\omega,\phi$.
This could be something ``discriminative'' like
\begin{align}
    \hat{W},\hat{J},\hat{\omega},\hat{\phi} &\leftarrow \argmin_{W,J,\omega,\phi} \sum_{i=1}^n\norm{\begin{bmatrix}\xi^{(W)}_i \\ \eta^{(W)}_i\end{bmatrix} - \begin{bmatrix}\xi^{(J)}_i \\ \eta^{(J)}_i\end{bmatrix}} ~.
\end{align}
Or this could be something ``generative'' like
\begin{align}
    \hat{W},\hat{J},\hat{\omega},\hat{\phi} &\leftarrow \argmin_{W,J,\omega,\phi} \sum_{i=1}^n\norm{\begin{bmatrix}x_i \\ v_i\end{bmatrix} - T_W^{-1}(\begin{bmatrix}\xi^{(J)}_i \\ \eta^{(J)}_i\end{bmatrix})} ~,
\end{align}
where $T_W^{-1}$ is the inverse of the map.
In principle the generative direction is more scientifically responsible, if we think of the $x_i, y_i$ as being ``data.''
Since our goal is $T_W$ and not (particularly) $T_W^{-1}$, which we choose depends on how easy it is to get inverses.\footnote{Presumably getting inverses is either very hard or else must be framed as a separate learning problem?}

\section{Three-dimensional problem}
How does this generalize to three dimensions?
Is it as simple as
\begin{align}
    \begin{bmatrix}\xi^{(J)}_{ki} \\ \eta^{(J)}_{ki}\end{bmatrix} &= \begin{bmatrix}\sqrt{J_k\,\omega_k^{-1}}\,\sin(\omega_k\,t_i+\phi_k) \\ \sqrt{J_k\,\omega_k}\,\cos(\omega_k\,t_i+\phi_k)\end{bmatrix} ~,
\end{align}
where $1\leq k\leq 3$?

\end{document}
