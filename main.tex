% Copyright 2025 the author(s).\

% style notes
% -----------
% - write such that physicists, mathematicians, and statisticians would all be happy.
% - all equations get numbers; no exceptions. Use the align environment.
% - the multiply operator is \,
% - etc.

\documentclass{article}
\usepackage[letterpaper]{geometry}
\usepackage{amsmath, amsfonts}

% typesetting
\setlength{\textheight}{8.70in}
\setlength{\textwidth}{5.00in}
\setlength{\oddsidemargin}{3.25in}\addtolength{\oddsidemargin}{-0.5\textwidth}
\setlength{\topmargin}{-0.40in}
\frenchspacing\sloppy\sloppypar\raggedbottom

% math macros
\newcommand{\dd}{\mathrm{d}}
\def \R    {{\mathbb R}}

\DeclareMathOperator*{\argmin}{argmin}
\newcommand{\norm}[1]{\left\lVert#1\right\rVert_2^2}

% units macros
\newcommand{\unit}[1]{\mathrm{#1}}
\newcommand{\m}{\unit{m}}
\newcommand{\s}{\unit{s}}
\newcommand{\ps}{\s^{-1}}
\newcommand{\mps}{\m\,\ps}
\newcommand{\mmps}{\m^2\,\ps}

\usepackage{xcolor}
\newcommand{\blue}[1] {{\color{blue}#1}}

\title{\bfseries%
A toy problem for symplectic maps}
\author{The participants of \textsl{Geometry@FI}}
\date{2025 January}

\begin{document}

\maketitle

\section{One-dimensional problem}
In the toy setup, we are given $n$ one-dimensional positions $x_i$ (units of $\m$) and one-dimensional velocities $v_i$ (units of $\mps$) corresponding to times $t_i$ (units of $\s$), with $1\leq i\leq n$.
These points are $n$ snapshots at times $t_i$ of a particle in phase space on an orbit.
Our goal is to find an area-preserving map $T_W$ from $x, v$ to new coordinates $\xi, \eta$ such that
\begin{align}
    \begin{bmatrix}\xi^{(W)}_i \\ \eta^{(W)}_i\end{bmatrix} &= T_W\left(\begin{bmatrix}x_i \\ v_i\end{bmatrix}\right) ~,
\end{align}
where the function $T_W$ is presumably extremely nonlinear in its inputs.
The new coordinates ought to lie (nearly) on an ellipse of the form
\begin{align}
    \begin{bmatrix}\xi^{(J)}_i \\ \eta^{(J)}_i\end{bmatrix} &= \begin{bmatrix}\sqrt{2\,J\,\omega^{-1}}\,\sin(\omega\,t_i+\phi) \\ \sqrt{2\,J\,\omega}\,\cos(\omega\,t_i+\phi)\end{bmatrix}
\end{align}
for some values of $J$ (units of $\mmps$), $\omega$ (units of $\ps$), and $\phi$ (dimensionless).\footnote{Side question: Do we need to have a $\phi$ variable? Or can we set $\phi=0$ without loss of generality?}
The area-preserving map will have parameters or weights $W$ (which, in practice, will be a large structured blob of numbers, some of which may have units).

We are going to solve this problem by minimizing a cost function with respect to the parameters $W,J,\omega,\phi$.
This could be something ``discriminative'' like
\begin{align}
    \hat{W},\hat{J},\hat{\omega},\hat{\phi} &\leftarrow \argmin_{W,J,\omega,\phi} \sum_{i=1}^n\norm{\begin{bmatrix}\xi^{(W)}_i \\ \eta^{(W)}_i\end{bmatrix} - \begin{bmatrix}\xi^{(J)}_i \\ \eta^{(J)}_i\end{bmatrix}} ~.
\end{align}
Or this could be something ``generative'' like
\begin{align}
    \hat{W},\hat{J},\hat{\omega},\hat{\phi} &\leftarrow \argmin_{W,J,\omega,\phi} \sum_{i=1}^n\norm{\begin{bmatrix}x_i \\ v_i\end{bmatrix} - T_W^{-1}(\begin{bmatrix}\xi^{(J)}_i \\ \eta^{(J)}_i\end{bmatrix})} ~,
\end{align}
where $T_W^{-1}$ is the inverse of the map.
In principle the generative direction is more scientifically responsible, if we think of the $x_i, y_i$ as being ``data.''
Since our goal is $T_W$ and not (particularly) $T_W^{-1}$, which we choose depends on how easy it is to get inverses.\footnote{The H\'enon maps presented by George K have simple inverses, so inversion may be very easy.}

\subsection{An Attempt to Show Hogg's Conjecture}

\textbf{Hogg's Conjecture.} Under some regularity conditions on the original orbits in the universe, there is a symplectic map $f : R^2 \to R^2$ such that the orbits become elliptical with a fixed angular frequency. 

Let's consider a single particle. 

To show this, let's write out a Hamiltonian to describe the original orbits
\begin{align}
    \frac{\dd x}{\dd t} &= - \frac{\partial \mathcal{H}}{\partial v}, \quad \frac{d v}{dt} = \frac{\partial \mathcal{H}}{\partial x}. 
\end{align}
We want to get a closed-form ODE in terms of the new phase parameters $f_1, f_2$ (not sure how the units come to play here). So, chain rule and plugging in the above time derivatives gives us 
\begin{align}
    \frac{\dd f_1}{\dd t} &= - \frac{\partial f_1}{\partial x} \frac{\partial \mathcal{H}}{\partial v} + \frac{\partial f_1}{\partial v} \frac{\partial \mathcal{H}}{\partial x} \quad
    \frac{\dd f_2}{\dd t} = - \frac{\partial f_2}{\partial x} \frac{\partial \mathcal{H}}{\partial v} + \frac{\partial f_2}{\partial v} \frac{\partial \mathcal{H}}{\partial x}.
\end{align}

The ODE for which the solution is an elliptical orbit with a fixed angular frequency is given by 
\begin{align}
    \frac{\dd}{\dd t} \begin{bmatrix} x \\ y \end{bmatrix} &= \begin{bmatrix} 0 & a \\ -b & 0 \end{bmatrix} \begin{bmatrix} x \\ y \end{bmatrix}, \quad a\,b > 0. 
\end{align}
This is a nice linear system and admits a solution:  $x(t)=\sqrt{a} \sin(\sqrt{a\,b}\,t + \phi) $, $y(t)= \sqrt{b} \cos(\sqrt{a\,b}\,t + \phi)$; the angular frequency is $\sqrt{a\,b}$ here. $\phi$ here gives us the initial condition, but if we have data from the complete orbit, it may be dropped (?). \footnote{Someone asks: Mmh, setting $a = 2\,J\,\omega^{-1}$ and $b = 2\,J\,\omega$ to match the above parameters gives us $\sqrt{a\,b}=2\,J$ for the angular frequency. Something seems a little bit off here? Hogg says: No that's correct; $J$ is the area of the ellipse divided by $2\pi$, so this should be correct.}

To ensure that $f_1, f_2$ follows an elliptical orbit, we would like to have
\begin{align}
    - \frac{\partial f_1}{\partial x} \frac{\partial \mathcal{H}}{\partial v} + \frac{\partial f_1}{\partial v} \frac{\partial \mathcal{H}}{\partial x} &= a\,f_2, \quad
     - \frac{\partial f_2}{\partial x} \frac{\partial \mathcal{H}}{\partial v} + \frac{\partial f_2}{\partial v} \frac{\partial \mathcal{H}}{\partial x} = -b\,f_1,
\end{align}
which can also be expressed using a Poisson bracket 
\begin{align}
    [f_1, \mathcal{H}] &= a f_2, \quad [f_2, \mathcal{H}] = -b\,f_1. 
\end{align}
This calculation shows that Hogg's conjecture holds iff we can find a symplectic map $f$ such that the above condition is satisfied for some real numbers $a, b$. 

\section{Three-dimensional problem}
How does this generalize to three dimensions?
Is it as simple as
\begin{align}
    \begin{bmatrix}\boldsymbol{\xi}^{(W)}_i \\ \boldsymbol{\eta}^{(W)}_i\end{bmatrix} &= T_W(\begin{bmatrix}\boldsymbol{x}_i \\ \boldsymbol{v}_i\end{bmatrix}) \\
    \begin{bmatrix}\xi^{(J)}_{ki} \\ \eta^{(J)}_{ki}\end{bmatrix} &= \begin{bmatrix}\sqrt{2\,J_k\,\omega_k^{-1}}\,\sin(\omega_k\,t_i+\phi_k) \\ \sqrt{2\,J_k\,\omega_k}\,\cos(\omega_k\,t_i+\phi_k)\end{bmatrix} ~,
\end{align}
where $\boldsymbol{x}, \boldsymbol{v}, \boldsymbol{\xi}, \boldsymbol{\eta}$ are now three-vectors (column vectors in $\mathbb{R}^{3\times 1}$ with the same units as $x, v, \xi, \eta$ above),
and $\xi_{ki}, \eta_{ki}$ are the $k$th components of the three-vectors (with $1\leq k\leq 3$)?

\section{One-dimensional non-toy problem}
Our recent work on the local dark-matter density \cite{horta} could be translated into this language as follows:

In the non-toy one-dimensional problem, we are given $n$ one-dimensional positions $x_i$ (units of $\m$), $n$ corresponding one-dimensional velocities $v_i$ (units of $\mps$, and $n$ corresponding $k$-element lists $Z_i$ (diverse units), with $1\leq i\leq n$.
In this case, the $n$ phase-space points are not points along one trajectory, but rather the positions and velocities of a set of stars.
The $n$ lists $Z_i$ are lists of stellar properties that are invariant with time (likely element abundances, but they could be birthdays or other invariant properties).
We seek a symplectic map $T_W$ from $x, v$ to $\xi, \eta$ such that
\begin{align}
    \begin{bmatrix}\xi_i \\ \eta_i\end{bmatrix} &= T(\begin{bmatrix}x_i \\ v_i\end{bmatrix}; W_T)\\
    \begin{bmatrix}\xi_i \\ \eta_i\end{bmatrix} &= \begin{bmatrix}\sqrt{2\,J_i}\,\sin(\theta_i) \\ \sqrt{2\,J_i}\,\cos(\theta_i)\end{bmatrix} \\
    J_i &= \frac{1}{2}\,\xi_i^2 + \frac{1}{2}\,\eta_i^2 ~,
\end{align}
where again $W_T$ is a big block of weights parameterizing the symplectic map,
there is now not just one invariant $J$ for all points but instead there is one $J_i$ for each star $i$,
and we have dropped the concept of frequency $\omega$.\footnote{Hogg is concerned about this move because then the units of $\xi,\eta$ become really weird; is that a problem? The mathematicians never think so.}

The assumption is that the distribution of the $Z_i$ depends on phase-space ($x, v$) location, but only through the one-dimensional invariants $J_i$ and not through the one-dimensional angles $\theta_i$.
That is, in addition to the map $T(\cdot;W_T)$ we will have to learn another function $\zeta(J;W_\zeta)$ that is used to predict the stellar properties $Z_i$, where $W_\zeta$ is another block of weights.
The objective function is then somehow
\begin{align}
    \hat{W_T},\hat{W_\zeta} &\leftarrow \argmin_{W_T, W_\zeta} \sum_{i=1}^n\norm{Z_i - \zeta(J_i; W_\zeta)} ~,
\end{align}
where the dependence of the $J_i$ on the $W_T$ is left implicit.
There is a better objective function that weights the $Z_i$ by their inverse variances, but that change is officially trivial.

\section{Three-dimensional non-toy problem}
HOGG: WRITE HERE.

\section{Architectures}

Given $\{ x_i, v_i, Q_i \}_{i=1}^n$, we want to learn latent $(J, \theta)$ so that $J$ is a sufficient statistic for $Q$

\textbf{George's method I.} 
\begin{enumerate}
    \item $(x, v) \overset{g}{\mapsto} J \overset{f}{\mapsto} Q$
    \item After finding $g, f$ (so that we know $J$), we learn a Hamiltonian $h(x, v)$ as a symplectic completion or take a HenonNetwork that takes $x, v$ to $(J, \theta)$ (where $J$ is known), in order to learn $\theta$. (Remark: if $J$ is given (or we think we learn it perfectly), then we can let $\bar{x} = J$ and use Parra's method)
    % (in Cartesian or polar coordinates)
    % \begin{align}
    %   J_i &= \frac{\partial h}{\partial v} \mid_{(x, v)}, \quad Q_i = \frac{\partial h}{\partial x} \mid_{(x, v)} \\
    %   J_i \cos \theta_i  &= \frac{\partial h}{\partial v} \mid_{(x, v)}, \quad Q_i \sin \theta_i = \frac{\partial h}{\partial x} \mid_{(x, v)}
    % \end{align}
    
\end{enumerate}

\textbf{George's method II.} 
Do George's method I with the two steps simultaneously.

\textbf{Lawrence's method.}

\begin{enumerate}
    \item $(x, v) \overset{\operatorname{symp}}{\mapsto} (J, \theta) $ where $Q \approx f(J)$.
    \item Regularization: to constrain $\theta$ to be the angle we want
\end{enumerate}


\section{Symmetries}
Hogg said for $x, v \in \R^d$:
\begin{itemize}
\item We may need to worry about the permutation symmetry among the $d$ coordinates
\item We do not need to worry about the rotation symmetry among the $d$ coordinates due to the toroidal geometry.
\end{itemize}

George said we may worry less, since the sympletic transformation inherits the symmetries from the beginning.

George asked: is our system energy-preserving (steady state)?  if we have a dissipative system, we might need to worry about changes in symmetries due to changes of energy. 

References \cite{duruisseaux}.

\bibliographystyle{plain}
\bibliography{symplectic}

\end{document}
